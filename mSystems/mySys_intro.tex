The desire to understand the factors that influence human health and cause diseases has always been one of the major driving forces of biological research. Modern high-throughput sequencing and bioinformatics tools provide a powerful means of understanding how the human microbiome contributes to health and its potential as a target for therapeutic interventions. High throughput methods for microbial 16S ribosomal RNA gene and WGS have now begun to reveal the composition of archaeal, bacterial, fungal and viral communities located both, in and on the human body. Biology has recently acquired new technological and conceptual tools to investigate, model and understand living organisms at the system level, thanks to the spectacular progress in quantitative techniques, large-scale measurement methods and the integration of experimental and computational approaches. Systems Biology has mostly been devoted to the study of well-characterized model organisms but, since the early days of the Human Genome Project it has become clear that applications of system-wide approaches to Human Biology would bring huge opportunities in Medicine. Here we present the imprints of disease in macroscopic properties of the system, by studying the temporal variability in the microbiome.