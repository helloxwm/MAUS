% !TEX TS-program = pdflatex
% !TEX encoding = UTF-8 Unicode

\documentclass[12pt,oneside,letterpaper]{article}

% release codenumber (update it here!)
\newcommand\releasecode{final}

%%% PACKAGES
\usepackage{lmodern} % Activa fuentes y macros de Latin Modern
\usepackage[T1]{fontenc} % set font encoding
\usepackage[utf8]{inputenc} % set input encoding (not needed with XeLaTeX)
\usepackage[english]{babel} 
\usepackage[expert, charter]{mathdesign} % Fuentes... disponibles conjuntos {charter|utopia}
%%\usepackage{calligra}
\usepackage[printwatermark]{xwatermark}
%\usepackage{graphicx}
\usepackage[usenames,dvipsnames,table]{xcolor}
%\usepackage{parskip}
%%\usepackage{float}
%\usepackage[nolists,tablesfirst, nomarkers]{endfloat}
%\usepackage{moreverb} % for verbatim output

% text layout
\usepackage[letterpaper]{geometry}
\geometry{textwidth=16.25cm} % 15.25cm for single-space, 16.25cm for double-space
\geometry{textheight=22.5cm} % 22cm for single-space, 22.5cm for double-space

% watermarks (time consuming!)
\newwatermark[allpages,color=PineGreen!4,angle=45,scale=7,xpos=-40,ypos=0]{submitted}
\newwatermark[allpages,color=PineGreen!12,angle=0,scale=1,xpos=0,ypos=137]{
\emph{mSystems} \LaTeX\ manuscript \releasecode\ submission}

% helps to keep figures from being orphaned on a page by themselves
\renewcommand{\topfraction}{0.85}
\renewcommand{\textfraction}{0.1}
%% Roman footnote numbering style to not collide with bibliographic cites: 
\renewcommand{\thefootnote}{\Roman{footnote}}

% line numbering
\usepackage[running,mathlines]{lineno}
\renewcommand\thelinenumber{\color{red}\arabic{linenumber}}
\linenumbers

% bold the 'Figure #' in the caption and separate it with a period
% Captions will be left justified
\usepackage[labelfont=bf,labelsep=period,font=small]{caption}
\captionsetup[table]{name=Supplementary Table S\!\!}

% review layout with double-spacing
\usepackage{setspace} 
\captionsetup{labelfont=bf,labelsep=period,font=doublespacing}

% cite package, to clean up citations in the main text. Do not remove.
\usepackage{cite}
\renewcommand\citeleft{(}
\renewcommand\citeright{)}
\renewcommand\citeform[1]{\textsl{#1}}

% Remove brackets from numbering in list of References
\renewcommand\refname{\large References}
\makeatletter
\renewcommand{\@biblabel}[1]{\quad#1.}
\makeatother

% Package authblk
\usepackage{authblk}
\renewcommand\Authands{ \& }
\renewcommand\Authfont{\normalsize \bf}
\renewcommand\Affilfont{\small \normalfont}

% notation
\usepackage{amsmath}

%% Roman footnote numbering style to not collide with bibliographic cites: 
\renewcommand{\thefootnote}{\Roman{footnote}}

% aux commands
\newcommand{\CC}[0]{\emph{cmplxcruncher}}
\newcommand{\task}[1]{\texttt{\bfseries\scshape\textcolor{MidnightBlue}{#1}}}
\newcommand{\un}[1]{\operatorname{#1}}
\newcommand{\unclassrate}[0]{\un{kpb/s/core}}
\newcommand{\todo}[1]{\texttt{\bfseries\textcolor{Orange}{#1}}}


% word count (Need --enable-write18 or --shell-escape)
\immediate\write18{texcount -opt=option.tc \jobname.tex > wordcount.aux} 
\newcommand\wordcount{\verbatiminput{wordcount.aux}}


%%% DOCUMENT %%%

\begin{document}

\title{
	\vspace*{0mm} % May help to include the corresponding author footnote in the title page
	\singlespacing
	\begin{flushleft}
		\texttt{\large Title:} \\
	\end{flushleft}
	\vspace*{2mm}
	Supplemental material of "Health and disease imprinted in the time variability of the human microbiome"\\
	\vspace*{6mm}
	\begin{flushleft}
		\texttt{\large Running title:} \\
	\end{flushleft}
	\vspace*{0mm}
	Supplemental material of "Microbiota, are you sick?"
	\vspace*{4mm}
	}

\doublespacing

\author[1,2,$*$]{Jose Manuel Martí}
\author[1,2,3,$*$]{Daniel Martínez-Martínez}
\author[2]{Manuel Peña}
\author[1,2]{César Gracia}
\author[1,3,4,5]{Amparo Latorre}
\author[1,3,4,5,\#]{Andrés Moya}
\author[1,2,\#]{Carlos P. Garay}

\affil[1]{Institute for Integrative Systems Biology (I2SysBio), 46980, Spain.}
\affil[2]{Instituto de Fisica Corpuscular, CSIC-UVEG, P.O.  22085, 46071, Valencia, Spain.}
\affil[3]{FISABIO, Avda de Catalunya, 21, 46020, Valencia, Spain.}
\affil[4]{Cavanilles Institute of Biodiversity and Evolutionary Biology, UVEG, 46980, Spain.}
\affil[5]{CIBER en Epidemiología y Salud Pública (CIBEResp), Madrid, Spain}

\date{}

\maketitle
\footnote[0]{$^*$ Equally contributed}
\footnote[0]{$^\#$ Corresponding authors: andres.moya@uv.es, penagaray@gmail.com}

\clearpage
\begin{table}
  \caption{Taylor parameters. Individuals with either animal-based (A) or plant-based (P) diets\cite{diet}. Previous to diet,  the population sampled is described by $\bar{V} = 0.09 \pm 0.05, \bar{\beta} = 0.77 \pm 0.04$.}

  \begin{center}
    \begin{tabular}{ccccccc}
	    \hline
		Metadata&V&$\beta$&$\bar{R}^2$&&V$_{st}$&$\beta_{st}$\\
		\hline
		A&$0.26 \pm 0.05$&$0.826 \pm 0.025$&$0.918$&&$3.1 \pm 0.9$&$1.2 \pm 0.6$\\
		A&$0.32 \pm 0.06$&$0.857 \pm 0.025$&$0.924$&&$4.4 \pm 1.1$&$2.0 \pm 0.6$\\
		A&$0.194 \pm 0.033$&$0.813 \pm 0.024$&$0.918$&&$1.9 \pm 0.6$&$0.9 \pm 0.6$\\
		A&$0.24 \pm 0.04$&$0.824 \pm 0.020$&$0.924$&&$2.7 \pm 0.7$&$1.2 \pm 0.5$\\
		A&$0.34 \pm 0.06$&$0.855 \pm 0.024$&$0.931$&&$4.7 \pm 1.1$&$1.9 \pm 0.6$\\
		A&$0.30 \pm 0.05$&$0.847 \pm 0.022$&$0.921$&&$3.9 \pm 1.0$&$1.7 \pm 0.5$\\
		A&$0.133 \pm 0.021$&$0.784 \pm 0.023$&$0.916$&&$0.7 \pm 0.4$&$0.2 \pm 0.6$\\
		A&$0.25 \pm 0.04$&$0.831 \pm 0.024$&$0.929$&&$3.0 \pm 0.8$&$1.4 \pm 0.6$\\
		\hline
		P&$0.23 \pm 0.05$&$0.804 \pm 0.035$&$0.885$&&$2.6 \pm 0.9$&$0.7 \pm 0.8$\\
		P&$0.097 \pm 0.018$&$0.705 \pm 0.031$&$0.891$&&$0.03 \pm 0.34$&$-1.6 \pm 0.7$\\
		P&$0.037 \pm 0.006$&$0.642 \pm 0.025$&$0.881$&&$-1.12 \pm 0.11$&$-3.1 \pm 0.6$\\
		P&$0.118 \pm 0.019$&$0.723 \pm 0.025$&$0.895$&&$0.4 \pm 0.4$&$-1.2 \pm 0.6$\\
		P&$0.17 \pm 0.04$&$0.78 \pm 0.04$&$0.842$&&$1.5 \pm 0.7$&$0.1 \pm 0.9$\\
		P&$0.123 \pm 0.020$&$0.757 \pm 0.026$&$0.914$&&$0.5 \pm 0.4$&$-0.4 \pm 0.6$\\
		P&$0.19 \pm 0.05$&$0.77 \pm 0.04$&$0.871$&&$1.8 \pm 0.9$&$-0.0 \pm 0.9$\\
		P&$0.121 \pm 0.020$&$0.736 \pm 0.027$&$0.921$&&$0.5 \pm 0.4$&$-0.9 \pm 0.6$\\
		P&$0.187 \pm 0.034$&$0.771 \pm 0.030$&$0.908$&&$1.8 \pm 0.7$&$-0.1 \pm 0.7$\\
		P&$0.097 \pm 0.015$&$0.735 \pm 0.025$&$0.922$&&$0.05 \pm 0.28$&$-0.9 \pm 0.6$\\
	   \hline
	   \hline
    \end{tabular}
  \end{center}
  \label{tab:diet}
\end{table}

\clearpage

\begin{table} 
\caption{Taylor parameters for individuals taking antibiotics\cite{antibiotic}. Prior to antibiotics intake, the population sampled is described by $\bar{V} = 0.12 \pm 0.05, \bar{\beta} = 0.75 \pm 0.04$.}
  \begin{center}
    \begin{tabular}{ccccccc}
	    \hline
		Metadata&V&$\beta$&$\bar{R}^2$&&V$_{st}$&$\beta_{st}$\\
		\hline
		Ab&$0.35 \pm 0.07$&$0.81 \pm 0.04$&$0.925$&&$4.3 \pm 1.4$&$1.3 \pm 0.9$\\
		Ab&$0.41 \pm 0.09$&$0.82 \pm 0.04$&$0.908$&&$5.6 \pm 1.8$&$1.6 \pm 0.9$\\
		Ab&$0.23 \pm 0.04$&$0.770 \pm 0.031$&$0.920$&&$2.1 \pm 0.8$&$0.5 \pm 0.7$\\
		Ab&$0.165 \pm 0.029$&$0.738 \pm 0.031$&$0.928$&&$0.9 \pm 0.6$&$-0.3 \pm 0.7$\\
		Ab&$0.34 \pm 0.06$&$0.812 \pm 0.032$&$0.936$&&$4.1 \pm 1.2$&$1.5 \pm 0.7$\\
		Ab&$0.26 \pm 0.05$&$0.798 \pm 0.033$&$0.931$&&$2.8 \pm 0.9$&$1.1 \pm 0.8$\\
	    \hline
	    \hline
    \end{tabular}
  \end{center}
  \label{tab:antibiotics}
\end{table}

\clearpage

\begin{table}
\caption{Taylor parameters for persons diagnosed with irritable bowel syndrome (IBS)\cite{IBS}. Healthy individuals sampled in this study are characterized by $\bar{V} = 0.134 \pm 0.009, \bar{\beta} = 0.691 \pm 0.025$.}
 
  \begin{center}
    \begin{tabular}{ccccccc}
	    \hline
		Metadata&V&$\beta$&$\bar{R}^2$&&V$_{st}$&$\beta_{st}$\\
		\hline
		IBS&$0.204 \pm 0.034$&$0.739 \pm 0.029$&$0.916$&&$7.6 \pm 3.7$&$1.9 \pm 1.2$\\
		IBS&$0.35 \pm 0.05$&$0.793 \pm 0.023$&$0.935$&&$23.1 \pm 5.9$&$4.0 \pm 0.9$\\
	     \hline
	     \hline
    \end{tabular}
  \end{center}
  \label{tab:IBS}
\end{table}

\clearpage

\begin{table} 
\caption{Taylor parameters for the healthy subject of the discordant twins\cite{kwashiorkor}. This table continues in Supplementary Table \ref{tab:DK}. The population of healthy twins is characterized by $\bar{V} = 0.25 \pm 0.10, \bar{\beta} = 0.863 \pm 0.028$.}
  \begin{center}
    \begin{tabular}{ccccccc}
	    \hline
		Metadata&V&$\beta$&$\bar{R}^2$&&V$_{st}$&$\beta_{st}$\\
		\hline
		DH&$0.27 \pm 0.04$&$0.835 \pm 0.016$&$0.925$&&$0.2 \pm 0.4$&$-1.0 \pm 0.6$\\
		DH&$0.36 \pm 0.06$&$0.858 \pm 0.015$&$0.929$&&$1.1 \pm 0.6$&$-0.2 \pm 0.5$\\
		DH&$0.35 \pm 0.06$&$0.859 \pm 0.014$&$0.926$&&$1.0 \pm 0.5$&$-0.1 \pm 0.5$\\
		DH&$0.25 \pm 0.04$&$0.829 \pm 0.014$&$0.911$&&$0.0 \pm 0.4$&$-1.2 \pm 0.5$\\
		DH&$0.30 \pm 0.05$&$0.844 \pm 0.014$&$0.920$&&$0.5 \pm 0.4$&$-0.7 \pm 0.5$\\
		DH&$0.29 \pm 0.05$&$0.850 \pm 0.016$&$0.915$&&$0.4 \pm 0.5$&$-0.5 \pm 0.5$\\
		DH&$0.28 \pm 0.05$&$0.848 \pm 0.016$&$0.921$&&$0.3 \pm 0.5$&$-0.5 \pm 0.6$\\
		DH&$0.35 \pm 0.07$&$0.861 \pm 0.017$&$0.918$&&$0.9 \pm 0.6$&$-0.0 \pm 0.6$\\
		DH&$0.31 \pm 0.04$&$0.833 \pm 0.012$&$0.916$&&$0.6 \pm 0.4$&$-1.1 \pm 0.4$\\
		DH&$0.33 \pm 0.05$&$0.843 \pm 0.013$&$0.925$&&$0.8 \pm 0.5$&$-0.7 \pm 0.5$\\
		DH&$0.31 \pm 0.05$&$0.852 \pm 0.014$&$0.925$&&$0.6 \pm 0.5$&$-0.4 \pm 0.5$\\
		DH&$0.31 \pm 0.05$&$0.853 \pm 0.015$&$0.930$&&$0.6 \pm 0.5$&$-0.4 \pm 0.5$\\
		DH&$0.203 \pm 0.033$&$0.815 \pm 0.015$&$0.907$&&$-0.44 \pm 0.32$&$-1.7 \pm 0.5$\\
		\hline
    \end{tabular}
  \end{center}
  \label{tab:DH}
\end{table}

\clearpage

\begin{table}
\caption{Taylor parameters for the kwashiorkor part of the discordant twins\cite{kwashiorkor}. This is a continuation of Supplementary Table \ref{tab:DH}. The population of healthy twins is characterized by $\bar{V} = 0.25 \pm 0.10, \bar{\beta} = 0.863 \pm 0.028$.}
   \begin{center}
    \begin{tabular}{ccccccc}
	    \hline
		Metadata&V&$\beta$&$\bar{R}^2$&&V$_{st}$&$\beta_{st}$\\
		\hline
		DK&$0.40 \pm 0.07$&$0.859 \pm 0.017$&$0.926$&&$1.5 \pm 0.7$&$-0.1 \pm 0.6$\\
		DK&$0.44 \pm 0.08$&$0.868 \pm 0.016$&$0.919$&&$1.8 \pm 0.8$&$0.2 \pm 0.6$\\
		DK&$0.196 \pm 0.031$&$0.819 \pm 0.014$&$0.916$&&$-0.50 \pm 0.30$&$-1.5 \pm 0.5$\\
		DK&$0.160 \pm 0.026$&$0.798 \pm 0.015$&$0.904$&&$-0.85 \pm 0.25$&$-2.3 \pm 0.5$\\
		DK&$0.30 \pm 0.05$&$0.845 \pm 0.014$&$0.924$&&$0.5 \pm 0.4$&$-0.6 \pm 0.5$\\
		DK&$0.23 \pm 0.04$&$0.834 \pm 0.014$&$0.908$&&$-0.1 \pm 0.4$&$-1.0 \pm 0.5$\\
		DK&$0.27 \pm 0.05$&$0.848 \pm 0.015$&$0.930$&&$0.2 \pm 0.4$&$-0.5 \pm 0.5$\\
		DK&$0.35 \pm 0.07$&$0.860 \pm 0.019$&$0.916$&&$1.0 \pm 0.7$&$-0.1 \pm 0.7$\\
		DK&$0.34 \pm 0.05$&$0.835 \pm 0.012$&$0.917$&&$0.9 \pm 0.5$&$-1.0 \pm 0.4$\\
		DK&$0.25 \pm 0.04$&$0.831 \pm 0.012$&$0.912$&&$0.0 \pm 0.4$&$-1.1 \pm 0.4$\\
		DK&$0.36 \pm 0.06$&$0.858 \pm 0.013$&$0.918$&&$1.1 \pm 0.5$&$-0.2 \pm 0.5$\\
		DK&$0.31 \pm 0.06$&$0.851 \pm 0.016$&$0.924$&&$0.6 \pm 0.6$&$-0.4 \pm 0.6$\\
		DK&$0.149 \pm 0.022$&$0.799 \pm 0.013$&$0.905$&&$-0.96 \pm 0.22$&$-2.2 \pm 0.5$\\
	    \hline
	    \hline
    \end{tabular}
  \end{center}
    \label{tab:DK}
\end{table}

\clearpage

\begin{table}
\caption{Taylor parameters for individuals with different degrees of overweight and obesity\cite{LEA}. Healthy people in this study, whom were not obese, are characterized by $\bar{V} = 0.19 \pm 0.06, \bar{\beta} = 0.806 \pm 0.034$.} 
  \begin{center}
    \begin{tabular}{ccccccc}
	    \hline
		Metadata&V&$\beta$&$\bar{R}^2$&&V$_{st}$&$\beta_{st}$\\
		\hline
		OW&$0.59 \pm 0.12$&$0.894 \pm 0.034$&$0.920$&&$6.6 \pm 2.0$&$2.6 \pm 1.0$\\
		OW&$0.22 \pm 0.04$&$0.830 \pm 0.030$&$0.904$&&$0.5 \pm 0.6$&$0.7 \pm 0.9$\\
		\hline
		OBI&$0.28 \pm 0.04$&$0.855 \pm 0.022$&$0.958$&&$1.5 \pm 0.6$&$1.4 \pm 0.6$\\
		OBI&$0.33 \pm 0.07$&$0.870 \pm 0.031$&$0.916$&&$2.4 \pm 1.1$&$1.9 \pm 0.9$\\
		\hline
		OBII&$0.223 \pm 0.032$&$0.823 \pm 0.023$&$0.938$&&$0.6 \pm 0.5$&$0.5 \pm 0.7$\\
		OBII&$0.208 \pm 0.029$&$0.844 \pm 0.022$&$0.935$&&$0.4 \pm 0.5$&$1.1 \pm 0.7$\\
		\hline
		OBIII&$0.34 \pm 0.05$&$0.855 \pm 0.025$&$0.943$&&$2.5 \pm 0.9$&$1.4 \pm 0.7$\\
		OBIII&$0.26 \pm 0.04$&$0.845 \pm 0.026$&$0.954$&&$1.1 \pm 0.7$&$1.2 \pm 0.8$\\
		OBIII&$0.33 \pm 0.06$&$0.870 \pm 0.027$&$0.908$&&$2.4 \pm 1.0$&$1.9 \pm 0.8$\\
		OBIII&$0.200 \pm 0.026$&$0.843 \pm 0.020$&$0.949$&&$0.2 \pm 0.4$&$1.1 \pm 0.6$\\
		OBIII&$0.30 \pm 0.05$&$0.846 \pm 0.026$&$0.929$&&$1.9 \pm 0.8$&$1.2 \pm 0.7$\\
		OBIII&$0.176 \pm 0.029$&$0.826 \pm 0.026$&$0.894$&&$-0.2 \pm 0.5$&$0.6 \pm 0.8$\\
		OBIII&$0.30 \pm 0.06$&$0.841 \pm 0.031$&$0.896$&&$1.8 \pm 0.9$&$1.0 \pm 0.9$\\
		OBIII&$0.28 \pm 0.04$&$0.857 \pm 0.025$&$0.941$&&$1.5 \pm 0.7$&$1.5 \pm 0.7$\\
		OBIII&$0.122 \pm 0.018$&$0.822 \pm 0.024$&$0.930$&&$-1.05 \pm 0.30$&$0.5 \pm 0.7$\\
		\hline
		OBIIId&$0.47 \pm 0.08$&$0.872 \pm 0.023$&$0.945$&&$4.7 \pm 1.3$&$1.9 \pm 0.7$\\
		OBIIId&$0.38 \pm 0.06$&$0.846 \pm 0.023$&$0.951$&&$3.2 \pm 1.0$&$1.2 \pm 0.7$\\
		OBIIId&$0.36 \pm 0.06$&$0.842 \pm 0.022$&$0.954$&&$2.9 \pm 0.9$&$1.1 \pm 0.6$\\
	    \hline
	    \hline
    \end{tabular}
  \end{center}
  \label{tab:LEA}
\end{table}

\clearpage

\begin{thebibliography}{9}
\bibitem{diet} {\bf David LA, Maurice CF, Carmody RN, Gootenberg DB, Button JE, Wolfe BE, Ling A V, Devlin AS, Varma Y, Fischbach MA, Biddinger SB, Dutton RJ, Turnbaugh PJ.} 2014. Diet rapidly and reproducibly alters the human gut microbiome. Nature {\bf 505}:559–63.
\bibitem{antibiotic} {\bf Dethlefsen L, Relman DA.} 2011. Incomplete recovery and individualized responses of the human distal gut microbiota to repeated antibiotic perturbation. Proc Natl Acad Sci {\bf108}:4554–61.
\bibitem{IBS} {\bf Durbán A, Abellán JJ, Jiménez-Hernández N, Artacho A, Garrigues V, Ortiz V, Ponce J, Latorre A, Moya A.} 2013. Instability of the faecal microbiota in diarrhoea-predominant irritable bowel syndrome. FEMS Microbiol Ecol {\bf 86}:581–589.
\bibitem{kwashiorkor} {\bf Smith MI, Yatsunenko T, Manary MJ, Trehan I, Mkakosya R, Cheng J, Kau AL, Rich SS, Concannon P, Mychaleckyj JC, Liu J, Houpt E, Li J V, Holmes E, Nicholson J, Knights D, Ursell LK, Knight R, Gordon JI.} 2013. Gut microbiomes of Malawian twin pairs discordant for kwashiorkor. Science {\bf 339}:548–54.
\bibitem{LEA} {\bf Faith JJ, Guruge JL, Charbonneau M, Subramanian S, Seedorf H, Goodman AL, Clemente JC, Knight R, Heath AC, Leibel RL, Rosenbaum M, Gordon JI.} 2013. The long-term stability of the human gut microbiota. Science {\bf 341}:1237439.
\end{thebibliography}
\end{document}


