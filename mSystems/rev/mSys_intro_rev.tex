The desire to understand the factors that influence human health and cause disease has always been one of the major driving forces of biological research. As evidence of the new "holobiont"  and "hologenome" concepts is increasing each day \cite{holo1, holo2}, research not only focuses on the human physiology but also on the microbial population that surrounds us. However, these concepts are still under debate \cite{holo3}. We are populated by a myriad of microorganisms that interact with us in several physiological processes such as the metabolism of bile acids \cite{bileacids}, of choline \cite{choline} and key-route metabolites, such as short-chain fatty acids \cite{scfa1, scfa2} which are also involved in immune system maturation \cite{scfa3, scfa4}. Human microbiota has been suggested to be closely related to diseases like type 2 diabetes \cite{diabetes2}, cardiovascular disease (CVD) \cite{CVD}, irritable bowel syndrome \cite{IBS}, Crohn's disease \cite{CD}, some affections like obesity \cite{ob1, ob2} and malnutrition \cite{nutr} as well as other multiple diseases \cite{Moya_trends}. Current studies reveal that gut microbiota also influences brain function and behaviour and is related to neurological disorders like Alzheimer's disease through the brain-gut-microbiome axis\cite{mind,AD}. Recently, even a mystifying and elusive condition which is hard to diagnose like chronic fatigue syndrome, which has often been suggested to be a psychosomatic disease, has been closely related to reduced diversity and altered composition of the gut microbiome\cite{CFS}. 

High throughput methods for microbial 16S ribosomal RNA gene and SMS (shotgun metagenomic sequencing) have now begun to reveal the composition of archaeal, bacterial, fungal and viral communities located both, in and on the human body. Modern high-throughput sequencing and bioinformatics tools provide a powerful means of understanding how the human microbiome contributes to health and its potential as a target for therapeutic interventions \cite{microb&health}. To define normal host-gut microbe interactions and how microbiota compositional changes can cause some diseases are important issues that still require scientific answers \cite{normal1, normal2, panthropology}.

Biology has recently acquired new technological and conceptual tools to investigate, model and understand living organisms at system level, thanks to spectacular progress in quantitative techniques, large-scale measurement methods and the integration of experimental and computational approaches. In particular, Systems Biology has made great efforts to reveal the general laws governing the complex behaviour of microbial communities \cite{sysbio&microb, msys1, metasysbio}, including a proposal suggesting they have universal dynamics \cite{uni_dynam}. Microbiota can be approached under the light of ecological theory which includes general principles like Taylor's law \cite{taylor} that relates the spatial or temporal variability of the population with its mean. This law, also known as fluctuation scale law, is ubiquitous in the natural world and can be found in several systems like random walks \cite{randomwalks}, stock markets \cite{economics1, economics2}, tree \cite{cohen_taylor} and animal populations \cite{taylor, animal1, animal2}, gene expression \cite{genexpress}, and the human genome \cite{genome}. Taylor's law has been applied to microbiota in a spatial way in the work of Zhang {\it et al.}, (2014) \cite{isme1}, where they show that this population tends to be an aggregated one rather than having a random distribution. Despite its ubiquity, it has only been studied in experimental settings \cite{cohen_bac, ramslayer} but has never been applied in follow-up studies on microbiota, even though major efforts have been made to infer the community structure from a dynamic point of view \cite{cobas, schloss, ravel}  

This paper presents the imprints of health status (healthy or disease) in the macroscopic properties of microbiota, by studying its temporal variability. We analyzed more than 35000 time series of taxa from the gut microbiome of 99 individuals obtained from publicly available high throughput sequencing data about different conditions: diseases, diets, obese status, antibiotic therapy and healthy individuals. Having seen that all the cases followed Taylor's law, we used this empirical fact to model how the relative abundances of taxa evolved over time thanks to the Langevin equation, in a similar way as the approach applied recently by Blumm {\it et al.} \cite{ranking}. We used this mathematical framework to explore the temporal stability of microbiota under different conditions in order to understand how this affected the healthy status of the subjects.