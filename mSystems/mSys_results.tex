We have analysed the microbiome temporal variability to extract global properties of the system. As fluctuations in total counts are plagued by systematic errors we worked on temporal variability of relative abundances for each taxon. Our first finding was that, in all cases, changes in relative abundances of taxa follow a universal pattern known as the fluctuation scaling law\cite{fs} or Taylor's  power law\cite{taylor}, i.e., microbiota of all detected taxa follows a power law dependence between mean relative abundance $x_i$ and dispersion $\sigma_i$, $\sigma_i  = V\cdot x_i^{\beta}$. While the law is universal, spanning six orders of magnitude in the observed relative abundances, the power law (or scaling) index $\beta$ and the variability $V$ (hereafter Taylor parameters) appear to be correlated with the stability of the community and the health status of the host, which is the main finding in this letter (see Figure 1). Taylor parameters describing the temporal variability of the gut microbiome in our sampled individuals are shown in ST1-6. Our results hint at a universal behaviour. Firstly, the variability (which corresponds to the maximum amplitude of fluctuations) is large, which suggests resilient capacity of the microbiota, and the scaling index is always smaller than one, which means that, more abundant taxa are less volatile than less abundant ones. Secondly, Taylor parameters for the microbiome of healthy individuals in different studies are compatible within estimated errors. This enables us to define the health zone in the Taylor parameter space. We can better visualize the results of individuals from different studies by standardizing their Taylor parameters, where standardization means that each parameter is subtracted by the mean value and divided by the standard deviation of the group of healthy individuals in the study (see Supplementary Section 12 and ST1-6). The zone of health and the standardized Taylor parameters for individuals whose gut microbiota is threatened (i.e., suffering from kwashiorkor, altered diet, antibiotics, IBS) is shown in Figure 1. Children developing kwashiorkor show smaller variability than their healthy twins. A meat/fish-based diet increases the variability significantly when compared to a plant-based diet. All other cases presented increased variability, which is particularly severe, and statistically significant at more than 95\% CL, for obese patients grade III on a diet, individuals taking antibiotics or IBS--diagnosed patients. A global property emerges from all worldwide data collected: Taylor parameters characterize the statistical behaviour of microbiome changes. We have verified that our conclusions are robust to systematic errors due to taxonomic assignment.

Taylor's power law has been explained in terms of various effects, all without general consensus. It can be shown to have its origin in a mathematical convergence similar to the central limit theorem, so virtually any statistical model designed to produce a Taylor law converge to a Tweedie distribution\cite{stat}, providing a mechanistic explanation based on the statistical theory of errors\cite{convergence1,convergence2,convergence3}. To unveil the generic mechanisms that drive different scenarios in the $\beta$--V space, we model the system by assuming that taxon relative abundance follows a Langevin equation with a deterministic term that captures the fitness of each taxon and a randomness term with Gaussian random noise\cite{ranking}. Both terms are modelled by power laws, with coefficients that can be interpreted as the taxon fitness $F_i$ and the variability $V$ (see Section 1 in supplemental material). When $V$ is sufficiently low, abundances are stable in time.  Differences in variability $V$ can induce a noise-induced phase transition in relative abundances of taxa. The temporal evolution of the probability of a taxon having abundance $x_i$ given its fitness is governed by the Fokker--Planck equation. The results of solving this equation show that  the stability is best captured by fitness F and amplitude of fluctuations V phase space (see Figure 2). 

The model predicts two phases for the gut microbiome: a stable phase with large variability that permits some changes in the relative abundances of taxa and an unstable phase with larger variability, above the phase transition, where the order of abundant taxa varies significantly with time. The microbiome of all healthy individuals was found to be in the stable phase, while the microbiome of several other individuals was shown to be in the unstable phase. In particular, individuals taking antibiotics and IBS--diagnosed patient P2 had the most severe symptoms. In this phase diagram, each microbiota state is represented by a point at its measured variability V and inferred fitness F. The model predicts high average fitness for all taxa, i.e., taxa are narrowly distributed in F. The fitness parameter has been chosen with different values for demonstrative purposes. Fitness is larger for the healthiest subjects and smaller for the IBS--diagnosed patients.  