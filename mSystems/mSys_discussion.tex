
The main feature of this work is to have shown that the microbiota, independently of its condition, follows the Taylor's law. We have seen that the value of beta in each case is always less than the unity (using standard deviation in the law), which is informing us about the community structure. This means that the most abundant elements in the population are less volatile to perturbations in relative terms than the less abundant. The biological explanation for this universal pattern is not clear. There are studies that prove that having this kind of exponents may be due to the negative interactions in the population \cite{kilpatrick}, while others have demonstrated that it may depend on reproductive correlation \cite{ballantyne}. Nevertheless, none of these explanations are enough when we are talking about microbiota as the reproduction term is diffuse, and the interactions between its components are numerous and complex \cite{joao, mehta}.

Model and temporal variation of V.

From the biological perspective, too many questions arise with this work. Knowing that microbiota has universal properties, one of the main relevant questions is which mechanisms are involved in maintaining the population structure in order to have always values of beta below 1...

Specifically, the analysis of the rank stability of the samples of healthy and IBS diagnosed patients studied in our lab\cite{IBS}, suggests that the presence of a \emph{rank stability island} among medium-ranked taxa could be an indicator of a healthy microbiota. 

Final paragraph to talk about limitations and future perspectives: can we model stability in the functional landscape? Community assembly for itself doesn't explain everything, we need to move forward and think about what can be happening in complex ecosystems as the human microbiota.