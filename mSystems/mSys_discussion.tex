We have quantitatively characterized whether the microbiota belongs to a healthy individual or a subject corresponding to an altered or pathological state (i.e., altered diet, antibiotic treatment, early gut development, diagnosed IBS). Deciphering the mechanisms of disease requires in depth knowledge of the underlying biological mechanisms. We describe here the macroscopic behavior of disease by a noise-induced phase transition with a control parameter that can be measured by the temporal variability of the microbiome. The microbiota of healthy individuals and of individuals with pathologies represent different phases separated by this noise-induced phase transition. Improved high-throughput sequencing of samples from individuals monitored over time and taxonomic assigning methods will provide a better distinction among pathologies or altered states of the microbiota.

Final paragraph to talk about limitations and future perspectives: can we model stability in the functional landscape? Community assembly for itself doesn't explain everything, we need to move forward and think about what can be happening in complex ecosystems as the human microbiota. 